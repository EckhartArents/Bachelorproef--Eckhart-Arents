%%=============================================================================
%% Samenvatting
%%=============================================================================

% TODO: De "abstract" of samenvatting is een kernachtige (~ 1 blz. voor een
% thesis) synthese van het document.
%
% Deze aspecten moeten zeker aan bod komen:
% - Context: waarom is dit werk belangrijk?
% - Nood: waarom moest dit onderzocht worden?
% - Taak: wat heb je precies gedaan?
% - Object: wat staat in dit document geschreven?
% - Resultaat: wat was het resultaat?
% - Conclusie: wat is/zijn de belangrijkste conclusie(s)?
% - Perspectief: blijven er nog vragen open die in de toekomst nog kunnen
%    onderzocht worden? Wat is een mogelijk vervolg voor jouw onderzoek?
%
% LET OP! Een samenvatting is GEEN voorwoord!

%%---------- Nederlandse samenvatting -----------------------------------------
%
% TODO: Als je je bachelorproef in het Engels schrijft, moet je eerst een
% Nederlandse samenvatting invoegen. Haal daarvoor onderstaande code uit
% commentaar.
% Wie zijn bachelorproef in het Nederlands schrijft, kan dit negeren, de inhoud
% wordt niet in het document ingevoegd.

\IfLanguageName{english}{%
\selectlanguage{dutch}
\chapter*{Samenvatting}
\lipsum[1-4]
\selectlanguage{english}
}{}

%%---------- Samenvatting -----------------------------------------------------
% De samenvatting in de hoofdtaal van het document

\chapter*{\IfLanguageName{dutch}{Samenvatting}{Abstract}}

Container virtualisatie is een recent populair geworden technologie die op dit moment niet behandeld word aan De Hogeschool Gent. Om dit aan te leren aan de Hogeschool lijkt Docker en de bijhorende technologieën de meest voor de hand liggende keuze. Maar in de wereld van container virtualisatie zijn er meet spellers betrokken. In dit onderzoek is nagegaan of dat Docker of een andere Technologie het best gehikt is om te gebruiken bij het les geven. Hiervoor werden eerst alle mogelijke alternatieven voor Docker opgezocht.  Van creatie van een container tot het online delen en meerde containers gelijktijdig te beheren.
Per gevonden technologie werd vervolgens achterhaald of dat ze kunnen worden gebruikt door een student om mee te leren. Deze meest geschikte technologieën werden praktisch uitgeprobeerd om zo een duidelijk idee te krijgen van hun gebruik. Hierdoor werd duidelijk dat Docker de meest zekere keuze is en dat Podman een goed alternatief kan zijn en dat er geen duidelijke voorkeur kan worden gegeven. Voor het delen van Containers is de Docker Hub voldoende voor een student en kan de GitHub Container Registry een aanvulling zijn of zelf een vervanging. De beste optie voor het beheren van meerdere containers is Kubernetes.
