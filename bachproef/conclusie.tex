%%=============================================================================
%% Conclusie
%%=============================================================================

\chapter{Conclusie}
\label{ch:conclusie}

% Trek een duidelijke conclusie, in de vorm van een antwoord op de
% onderzoeksvra(a)g(en). Wat was jouw bijdrage aan het onderzoeksdomein en
% hoe biedt dit meerwaarde aan het vakgebied/doelgroep? 
% Reflecteer kritisch over het resultaat. In Engelse teksten wordt deze sectie
% ``Discussion'' genoemd. Had je deze uitkomst verwacht? Zijn er zaken die nog
% niet duidelijk zijn?
% Heeft het onderzoek geleid tot nieuwe vragen die uitnodigen tot verder 
%onderzoek?
Het eerste wat we uit dit onderzoek kunnen besluiten is dat om te leren werken met containers er best gebruik gemaakt wordt van een VM. Hiervoor is een Linux VM met 4 GB ram geheugen, meer dan 20 GB harde schijfruimte, 2 CPU’s en een webbrowser voldoende. Ten tweede zijn er vijf container virtualisatie technologieën of services die geschikt zijn om te gebruiken. Voor zowel engines als repositories is er een keuze tussen twee goede alternatieven. Enkel voor orchestration is er een duidelijk beste keuze. 

Van alle engines die gebruikt kunnen worden zijn Docker en Podman de eenvoudigste om mee te werken. Deze twee vertonen ook een grote gelijkenis, dus zou het niet te moeilijk mogen zijn om van de ene over te schakelen naar de andere. Echter voor het beheren van meerdere containers samen is er een groter verschil. Docker en de Docker compose houden het vrij eenvoudig, terwijl de Podman pods meer kunnen maar ook complexer in gebruik zijn. Echter beide zijn in principe niet nodig als het verbinden van meerdere containers enkel via de orchestration software geregeld wordt.

Om de images online te bewaren is gebruikmaken van de Docker Hub het zekerst om mee werken en deze registry wordt ook standaard door beide engines gebruikt.. Doordat de GitHub Container Registry nog in bèta was tijdens dit onderzoek kan niet met zekerheid gezegd worden of dat deze beter is voor de student om zijn of haar eigen images te bewaren. Het werken met de GitHub Container Registry zou wel kunnen dienen om te tonen dat er meer opties mogelijk zijn om container images te bewaren dan enkel de Docker Hub, en kan ook helpen om een student meer bekend te maken met bijkomende functionaliteiten van GitHub. 

Voor orchestration is het duidelijk dat Kubernetes de beste oplossing is. Het heeft voor container virtualisatie de meeste mogelijkheden, wordt het meest gebruikt in het bedrijfsleven en heeft een dashboard die het leren werken met Kubernetes beduidend vergemakkelijkt. Om effectief te werken met een Kubernetes instantie is Minikube als intermediair ook het beste. Minikube kan zonder bijkomende configuratie met Docker als achterliggende engine werken. Ook biedt het met een paar extra stappen ondersteuning voor Podman als engine.

Andere container virtualisatie software en technologieën kunnen vaak meer dan Docker, Podman of Kubernetes, maar dit gaat gepaard met een mindere focus op het optimaal beheer van containers en vertoont daardoor een verhoogde complexiteit.

Dit onderzoek heeft zich vooral toegespitst op het lokaal kunnen werken met containers waardoor het gebruik van containers in een Cloud omgeving niet behandeld werd. Om te bepalen wie de beste Cloud service provider is om studenten te leren werken met de specifieke uitdagingen van containers in een Cloud omgeving is verder onderzoek nodig.



