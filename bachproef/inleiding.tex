%%=============================================================================
%% Inleiding
%%=============================================================================

\chapter{\IfLanguageName{dutch}{Inleiding}{Introduction}}
\label{ch:inleiding}

Alle applicaties draaien op hardware met een vorm van besturingssysteem. Echter dit kan voor verschillen zorgen bij het uitvoeren als dezelfde applicatie op een ander systeem uitgevoerd wordt. Om deze verschillen in te perken kan een applicatie uitgevoerd worden op een vorm van virtuele hardware. De techniek van een volledige virtuele machine bestaat dan ook al een tijd. Hiermee wordt een virtuele omgeving voorzien met daarin het besturingssysteem en de hardware die gewenst is. Een probleem dat hierbij opkomt is dat het volledig besturingssysteem niet altijd nodig is om een applicatie te draaien. Zeker als je in een virtuele omgeving maar één specifieke applicatie wilt draaien is de overhead van een besturingssysteem iets dat je kan missen. Om één applicatie te draaien in een virtuele omgeving die die applicatie enkel voorziet van wat het nodig heeft is container virtualisatie in het leven geroepen.
 
Het idee om in virtuele context een applicatie enkel te voorzien van wat het nodig heeft bestaat al lang, maar de recente populariteit van de container virtualisatie is voornamelijk te danken aan Docker. Docker slaagde erin het beheren en draaien van containers veel toegankelijker te maken dan de oudere manier waarmee Linux al werkte. Verder heeft Docker ook veel bijgedragen tot het vormen van een standaard voor containers, hoe ziet een container er uit en hoe gedraagt die zich. Deze standaard manier van het werken met containers en hun voordelen zorgden er ook voor dat ze veel gebruikt worden in een Cloudomgeving. Een container zal zich hetzelfde gedragen op elke server onafhankelijk van de onderliggende hardware. Bijkomend kunnen containers vlug bijgemaakt worden, zodat er bij een toename van het gebruik of het wegvallen van een applicatie het probleem snel opgelost geraakt. Iedereen die met het technische deel van de Cloud in aanraking komt zou dus best kennis hebben van containers en hoe met deze te werken.

\section{\IfLanguageName{dutch}{Probleemstelling}{Problem Statement}}
\label{sec:probleemstelling}

Binnen de opleiding toegepaste informatica komt de klassieke virtuele machine al aan bod, maar de Cloud en aanverwante technologieën komen nog maar beperkt voor. Container virtualisatie is ook iets dat zeer belangrijk is om mee te kunnen werken in een wereld waarbij je niet noodzakelijk de servers die je gebruikt fysiek bij jou hebt staan. De studenten kunnen voorbereiden op het werken met containers is dan ook meerwaarde voor de opleiding.

\section{\IfLanguageName{dutch}{Onderzoeksvraag}{Research question}}
\label{sec:onderzoeksvraag}

Concreet is de vraag die deze thesis tracht te beantwoorden of dat Docker voldoet om te dienen als inleiding tot het gebruik van containers of dat een andere aanbieder beter zou zijn. Een aanbieder die net zoals Docker iemand in staat stelt om lokaal op hun computer met containers te werken of een aanbieder die alles onmiddellijk in de Cloud doet.
% Wees zo concreet mogelijk bij het formuleren van je onderzoeksvraag. Een onderzoeksvraag is trouwens iets waar nog niemand op dit moment een antwoord heeft (voor zover je kan nagaan). Het opzoeken van bestaande informatie (bv. ``welke tools bestaan er voor deze toepassing?'') is dus geen onderzoeksvraag. Je kan de onderzoeksvraag verder specifiëren in deelvragen. Bv.~als je onderzoek gaat over performantiemetingen, dan 

\section{\IfLanguageName{dutch}{Onderzoeksdoelstelling}{Research objective}}
\label{sec:onderzoeksdoelstelling}

Uit dit onderzoek moet duidelijk worden dat Docker of een alternatief voldoet om gebruikt te worden als leermateriaal om met containers te werken. Daartoe werd een vergelijkende analyse van Docker en zijn alternatieven uitgevoerd.

Voor het aftoetsen van een mogelijk aanvaardbaar alternatief zijn er veel zaken die in rekening moeten worden gebracht worden, zeker voor het gebruik als lesmateriaal. Ten eerste moet het kunnen worden uitgevoerd worden op elk besturingssysteem. Ten tweede moet het uiteraard gebruiksvriendelijk genoeg zijn om binnen een redelijke hoeveelheid tijd besproken te worden. Verder mag het studenten ook geen geld kosten. Lokaal moet het volledig gratis of gedurende een aanvaardbare proefperiode bruikbaar zijn. In een Cloud omgeving moet het bruikbaar zijn binnen de perken van het gebruikskrediet op de Cloud service waarop de student recht zou hebben.
% Wat is het beoogde resultaat van je bachelorproef? Wat zijn de criteria voor succes? Beschrijf die zo concreet mogelijk. Gaat het bv. om een proof-of-concept, een prototype, een verslag met aanbevelingen, een vergelijkende studie, enz.

\section{\IfLanguageName{dutch}{Opzet van deze bachelorproef}{Structure of this bachelor thesis}}
\label{sec:opzet-bachelorproef}

% Het is gebruikelijk aan het einde van de inleiding een overzicht te
% geven van de opbouw van de rest van de tekst. Deze sectie bevat al een aanzet
% die je kan aanvullen/aanpassen in functie van je eigen tekst.

De rest van deze bachelorproef is als volgt opgebouwd:

In Hoofdstuk~\ref{ch:stand-van-zaken} wordt een overzicht gegeven van de stand van zaken binnen Container virtualisatie, op basis van een literatuurstudie en een onderzoek naar software aanbieders.

In Hoofdstuk~\ref{ch:methodologie} wordt de methodologie toegelicht voor het kiezen en hoe de software werd uitgetest. Dit wordt gevolgd door een stap voor stap beschreven uitwerking van de testen en de resultaten die hieruit bekomen werden. 

In Hoofdstuk~\ref{ch:conclusie},tenslotte, wordt de conclusie gegeven en een antwoord geformuleerd op de onderzoeksvragen. Daarbij wordt ook een aanzet gegeven voor toekomstig verder onderzoek binnen dit domein.