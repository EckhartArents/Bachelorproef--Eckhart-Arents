%%=============================================================================
%% Inleiding
%%=============================================================================

\chapter{\IfLanguageName{dutch}{Inleiding}{Introduction}}
\label{ch:inleiding}

% TODO: vooledig te herzien

Er is altijd al een nood geweest om software te kunnen draaien volledig onafhankelijk van de hardware en het besturingssysteem van een machine. De methodiek van een volledige virtual machine bestaat dan ook al een tijd. Een virtuele omgeving voorzie je het besturingssysteem van de hardware die wenst dat het heeft. Een probleem dat hierbij opkomt is dat het volledig besturingssysteem niet altijd nodig is om een applicatie te draaien.  Zeker als je in een virtueel een omgeving maar één specifieke applicatie wilt draaien is de overhead van een besturingssysteem iets dat je kan missen.  Om één applicatie te draaien in een virtueel omgeving die het enkel voorziet van wat het nodig heeft is container virtualisatie in het leven geroepen.
 
Het idee om in virtuele context een applicatie enkel toegang te geven tot wat het nodig heeft bestaat al lang, maar de recente populariteit van de container virtualisatie is voornamelijk te danken aan docker.  Docker slaagde erin het beheren en draaien van containers veel toegankelijker te maken dat de oudere manier waarmee Linux al werkte.  Verder heeft docker ook veel bijgedragen tot een standaard voor containers, hoe ziet een container er uit en hoe gedraagt het zich.  Deze standaard manier van het werken met containers en hun verticaliteit hielp ook bij het opzetten van de Cloud. Een container zal zich hetzelfde gedragen op elke server onafhankelijk van de onderliggende hardware. Bijkomend kunnen containers vlug bij gemaakt worden, zodanig dat als er een toename aan vraag is of er ergen een applicatie faalt het probleem snel opgelost geraakt. Iedereen die met het technische deel van de Cloud in aanraking komt zou dus best kennis hebben van containers en hoe met deze te werken.


\section{\IfLanguageName{dutch}{Probleemstelling}{Problem Statement}}
\label{sec:probleemstelling}

Binnen de opleiding toegepaste informatica wordt de klassieke virtuele machine al aan bod gebracht, maar de Cloud en verwante technologieën komen nog maar beperkt voor. Container virtualisatie is ook  iets dat zeer belangrijk is om mee te kunnen werken in een wereld waarbij je niet noodzakelijk de servers die je gebruikt fysiek bij jou hebt staan. De studenten kunnen voorbereiden op containers is dan ook meerwaarde voor de opleiding.

\section{\IfLanguageName{dutch}{Onderzoeksvraag}{Research question}}
\label{sec:onderzoeksvraag}

Concreet is de vraag van deze thesis of dat Docker voldoet om als inleiding tot het gebruik van containers of dat een andere aanbieder beter zou zijn. Een aanbieder die net zoals docker iemand in staat stelt om lokaal op hun computer met containers te werken of een aanbieder die alles onmiddellijk  in de Cloud doet.
% Wees zo concreet mogelijk bij het formuleren van je onderzoeksvraag. Een onderzoeksvraag is trouwens iets waar nog niemand op dit moment een antwoord heeft (voor zover je kan nagaan). Het opzoeken van bestaande informatie (bv. ``welke tools bestaan er voor deze toepassing?'') is dus geen onderzoeksvraag. Je kan de onderzoeksvraag verder specifiëren in deelvragen. Bv.~als je onderzoek gaat over performantiemetingen, dan 

\section{\IfLanguageName{dutch}{Onderzoeksdoelstelling}{Research objective}}
\label{sec:onderzoeksdoelstelling}

Uit dit onderzoek moet duidelijk zijn dat Docker of een alternatief voldoet om gebruikt te worden als leermateriaal om met containers te werken. Dit door zowel een vergelijkende analyse van Dockers en zijn alternatieven, als het opzetten van een proof of concept demonstratie met de meest belovende technologieën.

Voor het aftoetsen van een mogelijk aanvaardbaar alternatief zijn er veel zaken die in het in rekening moeten worden gebracht worden, zeker voor het fungeren als les materiaal. Ten eerste moet het kunnen worden uitgevoerd worden op elk besturingssysteem. Ten tweede moet het uiteraard gebruiksvriendelijk genoeg zijn om op een redelijke hoeveelheid tijd besproken te worden.  Verder mag het studenten ook geen geld kosten. Lokaal moet het volledig gratis of een aanvaardbare trial periode doenbaar zijn. In een Cloud omgeving moet het doenbaar zijn binnen de perken van wat krediet op de Cloud service waarop de student recht zou hebben. Tenslotte zou het effectief opzetten van de Drupal site met de technologie, gebruik makende van containers, niet langer dan 4 uur mogen duren.
% Wat is het beoogde resultaat van je bachelorproef? Wat zijn de criteria voor succes? Beschrijf die zo concreet mogelijk. Gaat het bv. om een proof-of-concept, een prototype, een verslag met aanbevelingen, een vergelijkende studie, enz.

\section{\IfLanguageName{dutch}{Opzet van deze bachelorproef}{Structure of this bachelor thesis}}
\label{sec:opzet-bachelorproef}

% Het is gebruikelijk aan het einde van de inleiding een overzicht te
% geven van de opbouw van de rest van de tekst. Deze sectie bevat al een aanzet
% die je kan aanvullen/aanpassen in functie van je eigen tekst.

De rest van deze bachelorproef is als volgt opgebouwd:

In Hoofdstuk~\ref{ch:stand-van-zaken} wordt een overzicht gegeven van de stand van zaken binnen het onderzoeksdomein, op basis van een literatuurstudie.

In Hoofdstuk~\ref{ch:methodologie} wordt de methodologie toegelicht en worden de gebruikte onderzoekstechnieken besproken om een antwoord te kunnen formuleren op de onderzoeksvragen.

% TODO: Vul hier aan voor je eigen hoofstukken, één of twee zinnen per hoofdstuk

In Hoofdstuk~\ref{ch:conclusie}, tenslotte, wordt de conclusie gegeven en een antwoord geformuleerd op de onderzoeksvragen. Daarbij wordt ook een aanzet gegeven voor toekomstig onderzoek binnen dit domein.